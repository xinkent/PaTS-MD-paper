Proteins often undergo large-amplitude conformational transitions when they exert their biological functions such as protein-folding, enzyme catalysis, and molecular recognition with ligand binding proteins. Elucidating these structural transitions mechanisms is essential for understanding the biological functions. Since experimentally investigating the dynamics of protein conformational transitions is difficult, most of the knowledge from conformational transitions in proteins is derived from molecular simulation techniques.

Molecular Dynamics (MD) simulations is a powerful approach to find the pathway of protein conformational transitions. It can provide structural information as a time series of atomic-level trajectories with femtosecond time resolution. However, it is still be difficult to observe the large motion changes relevant to important biological functions because the accessible time scale of a conventional simulation is often too shorter than the characteristic time scales of the biological functions.
Furthermore, since conformational transitions relevant to biological functions occur stochastically, it is not ensured that the long-time MD simulations surely capture these rare conformational transitions.
To tackle this problem, many enhanced conformational sampling methods have been proposed  

Parallel Cascade Selection Molecular Dynamics(PaCS-MD) \citep{harada2013parallel} is one of the methods to find the conformational transition pathway between initial structure and target structure when both structures are known {\it a priori}. In PaCS-MD, multiple short-time MD simulations are simply repeated until it reaches the structure sufficiently closer to the target structure.
Though PaCS-MD can fast generate the transition pathway to the target structure without adding any force bias, it has a critical problem that it can be trapped in stable local optima since PaCS-MD is a quite greedy algorithm. In other words, once PaCS-MD reaches a stable local free-energy minima, it cannot escape from it and search another way.

To solve this problem, in this study, we extend PaCS-Md by applying a tree search algorithm called UCT \citep{kocsis2006bandit} and propose a new method, Parallel Tree Search Molecular Dynamics (PaTS-MD). UCT is a tree search algorithm in which it select the node by balancing the trade-off between searching a node with best score and one which is not best now but is likely to become best if search more.
UCT can avoid being trapped in a stable local optima because nodes which are visited many times are penalized and are unlikely to be selected again. Therefore PaTS-MD can escape from a local minima even if it is trapped, and  can search broader space than PaCS-MD.

In the experiment, we compared PaCS-MD and PaTS-MD by applying both methods to folding study of mini-protein, chignolin and Trp-cage. In both experiment, we confirmed that PaTS-MD had more possibility to find the correct pathway.